\documentclass[a3paper,landscape]{article}
\usepackage[utf8]{inputenc}
\usepackage[russian]{babel}
\usepackage{geometry}
\usepackage{graphicx}
\usepackage{multicol}
\usepackage{qrcode}
\usepackage{xcolor}
\usepackage{titlesec}
\usepackage{enumitem}
\usepackage{rotating}

\geometry{margin=2cm}
\setlength{\columnsep}{2.5cm}

% Цветовая схема
\definecolor{primary}{RGB}{30, 70, 120}
\definecolor{secondary}{RGB}{70, 130, 180}
\definecolor{accent}{RGB}{220, 80, 60}

\titleformat{\section}{\large\bfseries\color{primary}}{\thesection}{1em}{}
\titleformat{\subsection}{\bfseries\color{secondary}}{\thesubsection}{1em}{}

\begin{document}

\pagestyle{empty}

\begin{center}
{\fontsize{36}{40}\selectfont\bfseries\color{primary} Настольные часы на ИВ-12 и STM32} \\
\vspace{0.8cm}
{\Large\bfseries Манжос Дмитрий, Аксёнов Арсений \quad Группа Б01-301 \quad 2025 год}
\end{center}

\vspace{1cm}

\begin{multicols}{2}

\section*{Цель проекта}
Разработать и создать настольные часы с винтажной индикацией на лампах ИВ-12, управляемые микроконтроллером STM32, сочетающие ретро-дизайн с современной элементной базой.

\section*{Основные компоненты}
\begin{itemize}[leftmargin=*,itemsep=0.5ex,topsep=0.5ex]
    \item Микроконтроллер: STM32F103C8T6 (Blue Pill)
    \item Индикаторы: 4×ИВ-12 (часы:минуты)
    \item Драйвер: ULN2003A
    \item Блок питания: DC-DC преобразователь 12V→25V/3.3V
    \item Корпус: 3D-печать (PLA)
\end{itemize}

\section*{Используемые технологии}
\begin{itemize}[leftmargin=*,itemsep=0.5ex,topsep=0.5ex]
    \item Проектирование схем: EasyEDA
    \item Разводка плат: FlatCAM
    \item 3D-моделирование: SolidWorks
    \item Программирование: STM32 HAL (C)
    \item Производство: ЧПУ-фрезеровка
\end{itemize}

\subsection*{Принцип работы}
\begin{enumerate}[leftmargin=*,itemsep=0.5ex,topsep=0.5ex]
    \item STM32 получает время от внутреннего RTC
    \item Формирует сигналы управления для драйвера ULN2003A
    \item Драйвер подаёт напряжение на соответствующие лампы ИВ-12
    \item Преобразователь обеспечивает необходимое питание
\end{enumerate}

\columnbreak

\section*{Финальный прототип}
\begin{center}
\includegraphics[width=0.8\linewidth,angle=180]{IMG_20250523_145455.jpg} \\
\vspace{0.3cm}
{\small Внешний вид собранного устройства}
\end{center}

\section*{Ключевые особенности}
\begin{itemize}[leftmargin=*,itemsep=0.5ex,topsep=0.5ex]
    \item Аутентичный ретро-стиль с лампами ИВ-12
    \item Открытая архитектура и документация
    \item Компактные габариты: 150×100×80 мм
    \item Автономное питание от 12В
    \item Возможность дальнейшей модернизации
\end{itemize}

\section*{Ресурсы проекта}
\begin{center}
\qrcode[height=3.2cm]{https://github.com/dmitrymanzhos/PCP} \\
\vspace{0.5cm}
\href{https://github.com/dmitrymanzhos/PCP}{\color{accent}\textbf{github.com/dmitrymanzhos/PCP}}
\end{center}

\end{multicols}

\vspace{0.8cm}
\begin{center}
\footnotesize Проект выполнен в рамках учебного курса. Требует доработки системы питания и оптимизации корпуса.
\end{center}

\end{document}