\documentclass[a4paper,12pt]{article}
\usepackage[utf8]{inputenc}
\usepackage[russian]{babel}
\usepackage{geometry}
\geometry{left=2.5cm,right=1.5cm,top=2cm,bottom=2cm}

\title{Отчёт по проекту настольных часов на ИВ-12 и STM32}
\author{Манжос Дмитрий, Аксёнов Арсений \\ Группа Б01-301}
\date{\today}

\begin{document}

\maketitle

\section{Цель проекта}
Спроектировать и изготовить настольные электронные часы с использованием микроконтроллера STM32, обеспечивающие достаточную точность хода с индикацией на лампах ИВ-12.

\section{Задачи}
\begin{itemize}
    \item Спроектировать печатную плату и блок питания
    \item Разработать прошивку с функционалом часов
    \item Изготовить корпус и плату
    \item Собрать устройство и провести тестирование
\end{itemize}

\section{Ход работы}
\subsection{Выбор компонентов}
Основные компоненты системы:
\begin{itemize}
    \item Индикаторные лампы ИВ-12
    \item Микроконтроллер STM32 Blue Pill
    \item Драйвер ULN2003a для управления лампами
\end{itemize}

\subsection{Прототипирование}
\begin{enumerate}
    \item Собрана принципиальная схема управления лампой на макетной плате
    \item Изучены особенности подключения и питания ламп ИВ-12
\end{enumerate}

\subsection{Разработка печатной платы}
\begin{itemize}
    \item Разведена и отфрезерована верхняя плата (первая версия)
    \item После доработок создан финальный вариант платы
\end{itemize}

\subsection{Корпус}
\begin{itemize}
    \item Разработана 3D-модель корпуса
    \item Корпус распечатан на 3D-принтере
\end{itemize}

\subsection{Программное обеспечение}
\begin{itemize}
    \item Реализована прошивка для управления часами
    \item Обеспечен базовый функционал отображения времени
\end{itemize}

\subsection{Сборка и тестирование}
\begin{itemize}
    \item Выполнена сборка и пайка компонентов
    \item Проведено тестирование системы
\end{itemize}

\section{Выявленные проблемы}
\begin{itemize}
    \item Некорректная работа преобразователя напряжения
    \item Несоответствие размеров корпуса
    \item Хрупкость тонких дорожек на верхней плате
    \item Необходимость навесного монтажа из-за сложности разводки
\end{itemize}

\section{Результаты}
\begin{itemize}
    \item Успешно спроектированы и изготовлены платы с интегрированным блоком питания
    \item Реализована рабочая прошивка для управления часами
    \item Проверен базовый функционал на макетной плате
    \item Выявлены проблемы, требующие доработки
\end{itemize}

\section{Заключение}
Проект реализован частично. Достигнуты основные задачи, однако выявленные проблемы требуют доработки схемотехнических решений, корпуса и программного обеспечения.

\end{document}