\documentclass[a4paper,12pt]{article}
\usepackage[utf8]{inputenc}
\usepackage[russian]{babel}
\usepackage{geometry}
\geometry{left=2cm,right=2cm,top=2cm,bottom=2cm}
\usepackage{enumitem}
\usepackage{hyperref}
\usepackage{graphicx}

\title{Пояснительная записка к проекту \\ "Настольные часы на ИВ-12 и STM32"}
\author{Манжос Дмитрий, Аксёнов Арсений \\ Группа Б01-301}
\date{23 мая 2025}

\begin{document}

\maketitle

\section{Состав проектной команды и распределение ролей}
\begin{itemize}[leftmargin=*]
    \item \textbf{Манжос Дмитрий}
    \begin{itemize}
        \item {Роль:} Проектирование верхней платы, фрезеровка плат, сборка и пайка компонентов, отчётность
        \item {Вклад:}
        \begin{itemize}
            \item Разводка и фрезеровка плат
            \item Подбор компонентов (ИВ-12, ULN2003a)
            \item сборка и пайка
        \end{itemize}
    \end{itemize}
    
    \item \textbf{Аксёнов Арсений}
    \begin{itemize}
        \item {Роль:} Программная часть, проектирование нижней части,  моделирование корпуса, сборка и пайка
        \item {Вклад:}
        \begin{itemize}
            \item Написание прошивки для STM32
            \item Разработка корпуса в Solidworks, его печать
            \item Сборка и пайка
        \end{itemize}
    \end{itemize}
\end{itemize}

\section{Причины выбора проекта}
\begin{itemize}
    \item Интерес к ретро-технологиям (винтажный эффект ламп ИВ-12)
    \item Практическое применение (полезное устройство для повседневного использования)
    \item Обучающий аспект (работа с STM32, схемотехникой и 3D-моделированием)
\end{itemize}

\section{Цель и задачи проекта}
\subsection{Цель}
Создать настольные часы на базе STM32 и индикаторных ламп ИВ-12.

\subsection{Задачи}
\begin{enumerate}
    \item Разработать принципиальную схему и печатную плату
    \item Реализовать прошивку с функционалом отображения времени
    \item Изготовить корпус методом 3D-печати
    \item Изготовить плату на фрезере
    \item Провести тестирование и отладку
\end{enumerate}

\section{Описание продукта}
\subsection{Аппаратная часть}
\begin{itemize}
    \item Микроконтроллер: STM32 Blue Pill
    \item Индикация: 4 x ИВ-12 (часы:минуты)
    \item Драйвер: ULN2003a для управления лампами
    \item Блок питания: DC-DC преобразователь (12V → 25V/3.3V/1.5V)
\end{itemize}

% \subsection{Программная часть}
% \begin{itemize}
%     \item Прошивка на C (STM32 HAL)
%     \item Алгоритм точного времени (RTC или таймеры)
% \end{itemize}

% \subsection{Корпус}
% \begin{itemize}
%     \item 3D-печатный, компактный
%     \item Отверстия для вентиляции
% \end{itemize}

\section{Процесс решения задач}
\subsection{Этапы работы}
\begin{enumerate}
    \item Аппаратный прототип:
    \begin{itemize}
        \item Сборка схемы на макетной плате
        \item Репозиторий: \url{https://github.com/dmitrymanzhos/PCP}
    \end{itemize}
    
    \item Печатная плата:
    \begin{itemize}
        \item Разводка в EasyEDA + FlatCAM
        \item Фрезеровка на станке Charly4U
    \end{itemize}
    
    \item Корпус:
    \begin{itemize}
        \item Модель в Solidworks
    \end{itemize}
\end{enumerate}

\subsection{Проблемы и решения}
\begin{itemize}
    \item {Проблема:} Некорректная работа DC-DC преобразователя \\
    
    \item {Проблема:} Хрупкость дорожек \\
\end{itemize}

\section{Анализ аналогов}
\subsection{Существующие аналоги}
\begin{itemize}
    \item Часы на Nixie-лампах:
    \begin{itemize}
        \item Дорогие компоненты (лампы IN-14)
    \end{itemize}
    
    \item Часы на светодиодах:
    \begin{itemize}
        \item Дешёвые, но менее эффектные
    \end{itemize}
\end{itemize}

\subsection{Отличия нашего проекта}
\begin{itemize}
    \item Использование доступных ламп ИВ-12
    \item Компактный корпус с современным дизайном
    \item Открытый исходный код и документация
\end{itemize}


\subsection{Результаты}
\begin{itemize}
    \item Изготовлены комплектующие, возникли проблемы, проект требует доработки
\end{itemize}

\section{Заключение}
Проект частично выполнен. Для финальной версии требуется доработка корпуса и стабилизация питания.

\end{document}